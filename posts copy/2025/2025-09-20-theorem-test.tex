% Add your research tags below (comma-separated, case-insensitive)
% Year is automatically added as a tag
% <TAGs>: 2025, theorem-test, mathematics, linear-algebra

\documentclass[11pt,letterpaper]{article}

\newcommand{\workingDate}{\textsc{2025 $|$ September $|$ 20}}
\newcommand{\userName}{Research Student}
\newcommand{\institution}{University} 

\usepackage{diary_base}
\usepackage{diary_math}
\usepackage{theorem_boxes}

\begin{document}
\href{run:2025-09-20-theorem-test.tex}{\Huge September 20} %##@@TitleDateString@@##

\section*{Testing Theorem Boxes}

This entry demonstrates the colorful theorem environments provided by theorem\_boxes.sty.

\begin{definition}
A \textbf{vector space} over a field $F$ is a set $V$ together with operations of addition and scalar multiplication that satisfy the vector space axioms.
\end{definition}

\begin{theorem}
Every finite-dimensional vector space has a basis.
\end{theorem}

\begin{proof}
Let $V$ be a finite-dimensional vector space. We can construct a basis by taking a maximal linearly independent set of vectors.
\end{proof}

\begin{example}
The set $\mathbb{R}^n$ with standard addition and scalar multiplication forms a vector space over $\mathbb{R}$.
\end{example}

\begin{remark}
The dimension of a vector space is well-defined and equals the cardinality of any basis.
\end{remark}

\begin{cbox}[red][Important Note]
This is a custom colored box that can be used for highlighting important information.
\end{cbox}

\begin{lemma}
If $T: V \to W$ is a linear transformation and $\{v_1, v_2, \ldots, v_n\}$ is linearly independent in $V$, then $\{T(v_1), T(v_2), \ldots, T(v_n)\}$ is linearly independent in $W$ if and only if $T$ is injective.
\end{lemma}

\bibliographystyle{apalike} 
\bibliography{reference}%##@@BibFileString@@##
\end{document}
