% Comprehensive test of all LaTeX commands in diary_commands.sty
%<TAGs>: test, commands, comprehensive
%<TOPICs>: latex-commands

\documentclass[11pt,letterpaper]{article}

\newcommand{\workingDate}{\textsc{2025 $|$ September $|$ 22}}
\newcommand{\userName}{Research Student}
\newcommand{\institution}{University} 

\usepackage{assets/styles/diary_base}        
\usepackage{assets/styles/diary_ctheorems}   
\usepackage{assets/styles/diary_commands}   

\begin{document}
\href{run:2025-09-22-command-test.tex}{\Huge September 22} 

\section{Comprehensive LaTeX Command Test}

This document tests all custom commands defined in \verb|diary_commands.sty| to verify the automatic command parsing system.

\subsection{Mathematical Sets and Spaces}
$$\RR^d, \CC^n, \NN, \PP, \E[X], \D$$
$$\P(\mathcal{A}), \H, \X, \B, \F, \M$$

\subsection{Statistical Functions}
$$\var(X), \Var(X), \cov(X,Y), \Cov(X,Y)$$
$$\corr(X,Y), \Corr(X,Y), \pr(A), \prob(B)$$
$$X \sim \normal(\mu, \sigma^2), \MSE, \KL(P \parallel Q)$$

\subsection{Math Operators}
$$x^* = \argmax_{x \in \RR} f(x)$$
$$y^* = \argmin_{y \in \CC} g(y)$$
$$\hat{\mu} = \median\{x_1, \ldots, x_n\}$$
$$\hat{x} = \mode\{x_1, \ldots, x_n\}$$

\subsection{Vectors and Bold Symbols}
$$\vv{x}, \vv{y}, \vv{\theta}, \vv{\beta}$$
$$\V{A}, \V{B}, \vx, \myb{z}$$

\subsection{Fractions and Derivatives}
$$\myf{a + b}{c + d} = \frac{\displaystyle a + b}{\displaystyle c + d}$$
$$\myp{f}{x} = \frac{\displaystyle \partial f}{\displaystyle \partial x}$$

\subsection{Special Symbols}
$$a \coloneq b + c$$

\subsection{Calculus and Analysis}
$$\dd f(x), \ind(A), \trace(A), \diag(A)$$

\subsection{Color Commands (for text)}
Some \red{red text}, \blue{blue text}, \green{green text}, \med{magenta text}, \gray{gray text}.

\todo{This is a todo item}

\subsection{Complex Mathematical Expression}
Combining multiple commands:
$$\E\left[\myf{\vv{x}^T \V{A} \vv{x}}{\trace(\V{B})}\right] = \argmax_{\vv{\theta} \in \RR^d} \prob\left(\vv{X} \sim \normal(\vv{\mu}, \V{\Sigma})\right)$$

Where we use the derivative:
$$\myp{L(\vv{\theta})}{\vv{\theta}} = \dd L(\vv{\theta})$$

And the indicator function:
$$\ind\left(\vv{x} \in \mathcal{X}\right) \cdot \var(Y)$$

\subsection{Delimited Commands Test}
Testing the \verb|\bb...\ee| commands:
\bb
\vv{x} = \argmax_{\vv{w}} \E\left[\myf{\vv{w}^T \vv{x}}{\trace(\V{K})}\right] \\
\text{subject to } \vv{w} \in \RR^d
\ee

\end{document}
